% Homework template for Information Theory and Statistical Learning
% by Xiangxiang Xu <xiangxiangxu.thu@gmail.com>
% LAST UPDATE: Oct 3, 2019
\documentclass[a4paper]{article}
\usepackage[T1]{fontenc}
\usepackage{amsmath, amssymb, amsthm}
% amsmath: equation*, amssymb: mathbb, amsthm: proof
\usepackage{moreenum}
\usepackage{mathtools}
\usepackage{url}
\usepackage{enumitem}
\usepackage{bm}
\usepackage{graphicx}
\usepackage{subcaption}
\usepackage{booktabs} % toprule
\usepackage[mathcal]{eucal}
\usepackage{dsfont}
\usepackage[numbered,framed]{matlab-prettifier}
%% Definitions for Information Theory & Statistical Learning
%% UPDATED: Sep 30, 2019 by Xiangxiang 
\newcommand{\theterm}{Fall 2020}

\newcommand{\thecoursenameshort}{\textsc{Information Theory and Statistical Learning}}
\newcommand{\thecoursename}{
Tsinghua-Berkeley Shenzhen Institute\\
%\vspace*{0.1in}
\thecoursenameshort
}

\newcommand{\courseheader}{
\vspace*{-1in}
\begin{center}
\thecoursename \\
\theterm
\vspace*{0.1in}
\hrule
\end{center}
}
\newcommand{\uc}{\underline{c}}    % c, vec
\newcommand{\uv}{\underline{v}}    % x, vec
\newcommand{\uw}{\underline{w}}    % w, vec
\newcommand{\ux}{\underline{x}}    % x, vec
\newcommand{\uy}{\underline{y}}    % y, vec
\newcommand{\uz}{\underline{z}}    % z, vec
\newcommand{\um}{\underline{m}}    % m, vec
\newcommand{\ut}{\underline{t}}    % t, vec

\newcommand{\bA}{\mathbf{A}}    % A, mat
\newcommand{\bI}{\mathbf{I}}    % A, mat
\newcommand{\bN}{\mathbf{N}}    % n, mat
\newcommand{\bT}{\mathbf{T}}    % T, mat
\newcommand{\bU}{\mathbf{U}}    % U, mat
\newcommand{\bV}{\mathbf{V}}    % V, mat
\newcommand{\bQ}{\mathbf{Q}}    % Q, mat
\newcommand{\bX}{\mathbf{X}}    % X, mat

\newcommand{\bc}{\bm{c}}    % c, vec
\newcommand{\be}{\bm{e}}    % e, vec
\newcommand{\bu}{\bm{u}}    % u, vec
\newcommand{\bv}{\bm{v}}    % v, vec
\newcommand{\bw}{\bm{w}}    % w, vec
\newcommand{\bt}{\bm{t}}    % t, vec
\newcommand{\bx}{\bm{x}}    % x, vec
\newcommand{\by}{\bm{y}}    % y, vec
\newcommand{\bz}{\bm{z}}    % z, vec

\newcommand{\phib}{\bm{\phi}}    % phi, vec
\newcommand{\psib}{\bm{\psi}}    % psi, vec
\newcommand{\dtm}{\mathbf{B}}    %
\newcommand{\dtmt}{\tilde{\dtm}}    %
\newcommand{\Ab}{\mathbf{A}}    % A mat
\newcommand{\Kb}{\mathbf{K}}    % K mat
\newcommand{\Ib}{\mathbf{I}}    % I mat



\newcommand{\rvby}{\bm{\mathsf{y}}}    % y, rv. vec
\newcommand{\rvbx}{\bm{\mathsf{x}}}    % x, rv. vec
% \newcommand{\bm}{\bm{m}}    % m, vec
\newcommand{\bzero}{\bm{0}}    % 0, vec

\newcommand{\balpha}{\bm{\alpha}}    % alpha, vec
\newcommand{\bphi}{\bm{\phi}}    % phi, vec
\newcommand{\bpsi}{\bm{\psi}}    % psi, vec
\newcommand{\bxi}{\bm{\xi}}    % xi, vec
\newcommand{\btheta}{\bm{\theta}}    % theta, vec
\newcommand{\bmu}{\bm{\mu}}    % mu, vec

\newcommand{\bLambda}{\bm{\Lambda}}    % Sigma, mat
\newcommand{\bSigma}{\bm{\Sigma}}    % Sigma, mat

\newcommand{\cF}{\mathcal{F}}  
\newcommand{\cL}{\mathcal{L}}  
\newcommand{\cX}{\mathcal{X}}  
\newcommand{\cY}{\mathcal{Y}}  

\newcommand{\rvx}{\mathsf{x}}    % x, r.v.
\newcommand{\rvy}{\mathsf{y}}    % y, r.v.
\newcommand{\rvz}{\mathsf{z}}    % z, r.v.
\newcommand{\rvw}{\mathsf{w}}    % w, r.v.
\newcommand{\rvv}{\mathsf{v}}    % v, r.v.
\newcommand{\rvm}{\mathsf{m}}    % m, r.v.
\newcommand{\rvt}{\mathsf{t}}    % t, r.v.
\newcommand{\rvH}{\mathsf{H}}    % H, r.v.
\newcommand{\urvx}{\underline{\mathsf{x}}}    % x, r.v. vec
\newcommand{\urvy}{\underline{\mathsf{y}}}    % y, r.v. vec
\newcommand{\urvz}{\underline{\mathsf{z}}}    % z, r.v. vec
\newcommand{\urvw}{\underline{\mathsf{w}}}    % w, r.v. vec
\newcommand{\urvt}{\underline{\mathsf{t}}}    % t, r.v. vec


\newcommand{\defeq}{\triangleq} %\coloneqq
\newcommand{\reals}{\mathbb{R}}
\newcommand{\T}{\mathrm{T}}    % transpose
\newcommand{\F}{\mathrm{F}}    % Frobenius
\newcommand{\BLS}{\mathrm{BLS}}    % BLS
\newcommand{\LLS}{\mathrm{LLS}}    % LLS
\newcommand{\MVU}{\mathrm{MVU}}    % MVU
\newcommand{\dd}{\mathrm{d}}  

\DeclareMathOperator*{\maximize}{maximize}    % maximize
\DeclareMathOperator*{\minimize}{minimize}    % minimize
\newcommand{\st}{\mathrm{subject~to}}    % minimize

% \newcommand{\E}[1]{\mathbb{E}\left[{#1}\right]}
% \newcommand{\Prob}[1]{\mathbb{P}\left({#1}\right)}
\DeclareMathOperator*{\argmax}{arg\,max}
\DeclareMathOperator*{\argmin}{arg\,min}
\DeclareMathOperator*{\argsup}{arg\,sup}
\DeclareMathOperator*{\arginf}{arg\,inf}
\DeclareMathOperator{\diag}{diag}
\DeclareMathOperator{\tr}{tr}
\DeclareMathOperator{\Cov}{Cov}
\DeclareMathOperator{\var}{var}
\DeclareMathOperator{\cov}{cov}
\DeclareMathOperator{\MSE}{MSE}
\DeclareMathOperator{\1}{\mathds{1}} % dsfont required
\DeclareMathOperator{\E}{\mathbb{E}}
\DeclareMathOperator{\Prob}{\mathbb{P}}
\DeclareMathOperator{\im}{im}
\DeclareMathOperator{\rank}{rank}
\DeclareMathOperator{\Bern}{Bern}
\DeclareMathOperator{\Binom}{Binom}

\newcommand\independent{\protect\mathpalette{\protect\independenT}{\perp}}
\def\independenT#1#2{\mathrel{\rlap{$#1#2$}\mkern2mu{#1#2}}}

%%% Local Variables:
%%% mode: latex
%%% TeX-master: "ithw"
%%% End:


\lstset{
  style              = Matlab-editor,
  captionpos         =b,
  basicstyle         = \mlttfamily,
  escapechar         = ",
  mlshowsectionrules = true,
}
\begin{document}
\courseheader

\newcounter{hwcnt}
\setcounter{hwcnt}{4} % set to the times of Homework

\begin{center}
  \underline{\bf Homework \thehwcnt} \\
\end{center}
\begin{flushleft}
  \textcolor{gray}{Hanmo Chen}\hfill
  \today
\end{flushleft}
\hrule

\vspace{2em}
\setlist[enumerate,1]{label=\thehwcnt.\arabic*.}
\setlist[enumerate,2]{label=(\alph*)}
\setlist[enumerate,3]{label=\roman*.}
\setlist[enumerate,4]{label=\greek*)}

\flushleft
\rule{\textwidth}{1pt}
\begin{itemize}
\item {\bf Acknowledgments: \/ For Problem 1, I refer to \url{ https://en.wikipedia.org/wiki/Incomplete_gamma_function} for Incomplete Gamma function} 
  \textcolor{gray}{None}
\item {\bf Collaborators: \/}
  \textcolor{gray}{I finish this homework by myself.} 
\item  \emph{I certify that all solutions are entirely in my words and that I have not looked at another student's solutions. I have credited all external sources in this write up.}
  \framebox[\linewidth]{\rule{0pt}{10pt}\textcolor{gray}{\large Hanmo Chen}}
\end{itemize}
\rule{\textwidth}{1pt}


\vspace{2em}



\begin{enumerate}
  \setlength{\itemsep}{3\parskip}

\item \begin{enumerate}
\item 

  Because $\mathrm{x}_i \overset{i.i.d.}{\sim} \mathcal{N}(0,\sigma^2)$, $\mathrm{y} = \sum\limits_{i=1}^{n} \frac{\mathrm{x}_i^2}{\sigma^2} \sim \chi^2_{n}$. $\mathbb{P}(Y\geqslant n\alpha^2 /\sigma^2 ) = \frac{\Gamma(\frac{n}{2},\frac{n\alpha^2}{2\sigma^2})}{\Gamma(\frac{n}{2})}$ where $\Gamma(s,x)$ denotes the upper Incomplete Gamma function.

So 

\begin{equation}
  \begin{aligned}
    -\frac{1}{n} \log \mathbb{P}(\frac{1}{n} \sum_{i=1}^n \mathrm{x}_i^2 \geqslant \alpha^2) =  -\frac{1}{n} \log \mathbb{P}(Y\geqslant \frac{n\alpha^2}{\sigma^2}) \\ = -\frac{1}{n} \log \frac{\Gamma(\frac{n}{2},\frac{n\alpha^2}{2\sigma^2})}{\Gamma(\frac{n}{2})}
  \end{aligned}
\end{equation}

To find the asymptotic property, using Sanov's theorem,
\begin{equation}
  \begin{aligned}
    \lim_{n \to \infty}-\frac{1}{n} \log \mathbb{P}(\frac{1}{n} \sum_{i=1}^n \mathrm{x}_i^2 \geqslant \alpha^2) = \inf_{\mathbb{E}_P[X^2] \geqslant \alpha^2} D(P\| \mathcal{N}(0,\sigma^2))
  \end{aligned}
\end{equation}
%   \item 

Suppose the distribution $P$ has pdf $f(x)$, it can be seen as an optimization problem with constraints, that is,

\begin{equation}
  \begin{aligned}
    &\min \quad & D(P\| \mathcal{N}(0,\sigma^2)) = \int f(x) \log{\frac{f(x)}{\frac{1}{\sqrt{2\pi \sigma^2}}e^{-\frac{x^2}{2\sigma^2}}} } \\
    & \text{s.t.} & \int f(x) x^2  \geqslant \alpha^2 \\
    & &\int f(x)= 1
  \end{aligned}
\end{equation}

Define 

\begin{equation}
  J(f) = \int f(x) \log{\frac{f(x)}{\frac{1}{\sqrt{2\pi \sigma^2}}e^{-\frac{x^2}{2\sigma^2}}} } + \lambda (\int f(x) x^2  - \alpha^2) + \mu(\int f(x) -1)
\end{equation}


And let $\frac{\partial J}{\partial f} = 0$ we have 

\begin{equation}
  \frac{\partial J}{\partial f} = \log f(x)  + \lambda x^2 + \mu = 0
\end{equation}

So $f(x) = \exp^{-\mu -\lambda x^2}$, which is normal distribution and satisfies $\mathbb{E}[X^2] \geqslant \alpha^2$. So $P^* = \mathcal{N}(0,\alpha^2)$ and 

\begin{equation}
  \begin{aligned}
    \lim_{n \to \infty}-\frac{1}{n} \log \mathbb{P}(\frac{1}{n} \sum_{i=1}^n \mathrm{x}_i^2 \geqslant \alpha^2) & =  D(P^*\| \mathcal{N}(0,\sigma^2))  \\& = \int_{\mathbb{R}} f(x) \log{\frac{\frac{1}{\sqrt{2\pi \alpha^2}}e^{-\frac{x^2}{2\alpha^2}}}{\frac{1}{\sqrt{2\pi \sigma^2}}e^{-\frac{x^2}{2\sigma^2}}} dx} \\
    & = \ln \frac{\sigma}{\alpha} + \frac{1}{2}(\frac{\alpha^2}{\sigma^2}-1)
  \end{aligned}
\end{equation}

\item Using the conclusion from (a), $P^* = \mathcal{N}(0,\alpha^2)$.

\end{enumerate}


\item \begin{enumerate}
  \item To prove the following lemma, 

\begin{equation}
  \left(\frac{n}{e}\right)^{n} \leqslant n ! \leqslant n\left(\frac{n}{e}\right)^{n}
\end{equation}

Which is equivalent to,

\begin{equation}
  n\ln n - n \leqslant \ln(n!) \leqslant (n+1)\ln n - n
\end{equation}

For the left part, notice that $\ln(1+\frac 1 {i})< \frac{1}{i}$ for $i\geqslant 1$ which leads to,

\begin{equation}
  (i+1) \ln(i+1) - i\ln i -1 < \ln(i+1)
\end{equation}

Sum for $i=1,2,\cdots,n-1$, we have 

\begin{equation}
  n\ln n - (n-1) < \sum_{i=1}^{n-1} \ln(i+1) = \ln (n!)
\end{equation}

For the right part, it holds only when $n\geqslant 7$. It is easy to check $n =7 $. So $\ln(7!) \leqslant 8\ln 7 - 7$

And for $n \geqslant 8$, because $\ln (1+x) > \frac{x}{x+1}$ for $x>0$, $\ln (1+ \frac 1 i ) > \frac{1}{i+1}$, $\ln i < (i+1)\ln (i+1) - i \ln i -1$ . 

Sum for $i=7,\cdots, n-1$

\begin{equation}
  \ln(6!)  + \sum_{i=7}^{n-1} \ln i + \ln n < 7\ln 7 - 7  + n\ln n + - 7\ln  7 + \ln n  - (n-7) = (n+1) \ln n - n
\end{equation}

So 

\begin{equation}
  \left(\frac{n}{e}\right)^{n} \leqslant n ! \leqslant n\left(\frac{n}{e}\right)^{n}
\end{equation}

\item From (a) we have that as $n \to \infty$,
\begin{equation}
  \frac{\ln (n!)}{n} \sim \ln \frac{n}{e}
\end{equation}

Therefore
\begin{equation}
  \begin{aligned}
    \lim _{n \to \infty} \frac{1}{n} \log \binom{n}{k} & = \lim _{n \to \infty} \frac{1}{n} \log \frac{n!}{k!(n-k)!}  \\ & = \lim _{n \to \infty} \log \frac{n}{e} - p \log \frac{pn}{e} - (1-p) \log \frac{(1-p)n}{e} \\
    & = -p\log p - (1-p) \log (1-p) = H(p)
  \end{aligned}
\end{equation}

Another explanation using Sanov's theorem, suppose $X_1,X_2,\cdots,X_n \overset{i.i.d}{\sim} Bernoulli(\frac{1}{2})$ consider 

\begin{equation}
  \lim_{n\to \infty} -\frac{1}{n} \log \mathbb{P}(\frac{1}{n} \sum_{i=1}^n X_i= p) 
\end{equation}

On the one hand, $\sum_{i=1}^n X_i \sim Binomial(n,\frac 1 2)$, so 

\begin{equation}
  \lim_{n\to \infty} -\frac{1}{n} \log \mathbb{P}(\frac{1}{n} \sum_{i=1}^n X_i= p) = \lim_{n\to \infty} -\frac{1}{n} \log \left[\binom{n}{k} \frac 1 {2^n}\right]  = 1 - \lim_{n\to \infty} \frac{1}{n} \log \binom{n}{k}
\end{equation}

On the other hand, using Sanov's theorem,

\begin{equation}
  \begin{aligned}
    \lim_{n\to \infty} -\frac{1}{n} \log \mathbb{P}(\frac{1}{n} \sum_{i=1}^n X_i= p)   & = D(Bernoulli(p) \| Bernoulli(\frac 1 2)) \\ & = p\log 2p + (1-p) \log(2(1-p))  \\ & = 1 + p\log p  + (1-p) \log (1-p)  \\ & = 1- H(p)
  \end{aligned}
\end{equation}

Also we can get 

\begin{equation}
  \lim_{n\to \infty} \frac{1}{n} \log \binom{n}{k} = H(p)
\end{equation}


Using the same way but using categorical and multinomial distribution instead of Bernoulli and  bionomial distribution.

\begin{equation}
  \lim _{n \rightarrow \infty} \frac{1}{n} \log \left(\begin{array}{c}n \\ \left\lfloor n p_{1}\right\rfloor\left\lfloor n p_{2}\right\rfloor \cdots\left\lfloor n p_{m-1}\right\rfloor\left(n-\sum_{i=1}^{m-1}\left\lfloor n p_{i}\right\rfloor\right)\end{array}\right) = -\sum_{i=1}^m p_i \log p_i
\end{equation}

where $\sum_{i=1}^m p_i = 1$.

\end{enumerate}

\item 
\end{enumerate}
  

  
\end{document}
%%% Local Variables:
%%% mode: latex
%%% TeX-master: t
%%% End:
