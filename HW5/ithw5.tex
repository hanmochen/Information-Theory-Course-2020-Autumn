% Homework template for Information Theory and Statistical Learning
% by Xiangxiang Xu <xiangxiangxu.thu@gmail.com>
% LAST UPDATE: Oct 3, 2019
\documentclass[a4paper]{article}
\usepackage[T1]{fontenc}
\usepackage{amsmath, amssymb, amsthm}
% amsmath: equation*, amssymb: mathbb, amsthm: proof
\usepackage{moreenum}
\usepackage{mathtools}
\usepackage{url}
\usepackage{enumitem}
\usepackage{bm}
\usepackage{graphicx}
\usepackage{subcaption}
\usepackage{booktabs} % toprule
\usepackage[mathcal]{eucal}
\usepackage{dsfont}
\usepackage[numbered,framed]{matlab-prettifier}
%% Definitions for Information Theory & Statistical Learning
%% UPDATED: Sep 30, 2019 by Xiangxiang 
\newcommand{\theterm}{Fall 2020}

\newcommand{\thecoursenameshort}{\textsc{Information Theory and Statistical Learning}}
\newcommand{\thecoursename}{
Tsinghua-Berkeley Shenzhen Institute\\
%\vspace*{0.1in}
\thecoursenameshort
}

\newcommand{\courseheader}{
\vspace*{-1in}
\begin{center}
\thecoursename \\
\theterm
\vspace*{0.1in}
\hrule
\end{center}
}
\newcommand{\uc}{\underline{c}}    % c, vec
\newcommand{\uv}{\underline{v}}    % x, vec
\newcommand{\uw}{\underline{w}}    % w, vec
\newcommand{\ux}{\underline{x}}    % x, vec
\newcommand{\uy}{\underline{y}}    % y, vec
\newcommand{\uz}{\underline{z}}    % z, vec
\newcommand{\um}{\underline{m}}    % m, vec
\newcommand{\ut}{\underline{t}}    % t, vec

\newcommand{\bA}{\mathbf{A}}    % A, mat
\newcommand{\bI}{\mathbf{I}}    % A, mat
\newcommand{\bN}{\mathbf{N}}    % n, mat
\newcommand{\bT}{\mathbf{T}}    % T, mat
\newcommand{\bU}{\mathbf{U}}    % U, mat
\newcommand{\bV}{\mathbf{V}}    % V, mat
\newcommand{\bQ}{\mathbf{Q}}    % Q, mat
\newcommand{\bX}{\mathbf{X}}    % X, mat

\newcommand{\bc}{\bm{c}}    % c, vec
\newcommand{\be}{\bm{e}}    % e, vec
\newcommand{\bu}{\bm{u}}    % u, vec
\newcommand{\bv}{\bm{v}}    % v, vec
\newcommand{\bw}{\bm{w}}    % w, vec
\newcommand{\bt}{\bm{t}}    % t, vec
\newcommand{\bx}{\bm{x}}    % x, vec
\newcommand{\by}{\bm{y}}    % y, vec
\newcommand{\bz}{\bm{z}}    % z, vec

\newcommand{\phib}{\bm{\phi}}    % phi, vec
\newcommand{\psib}{\bm{\psi}}    % psi, vec
\newcommand{\dtm}{\mathbf{B}}    %
\newcommand{\dtmt}{\tilde{\dtm}}    %
\newcommand{\Ab}{\mathbf{A}}    % A mat
\newcommand{\Kb}{\mathbf{K}}    % K mat
\newcommand{\Ib}{\mathbf{I}}    % I mat



\newcommand{\rvby}{\bm{\mathsf{y}}}    % y, rv. vec
\newcommand{\rvbx}{\bm{\mathsf{x}}}    % x, rv. vec
% \newcommand{\bm}{\bm{m}}    % m, vec
\newcommand{\bzero}{\bm{0}}    % 0, vec

\newcommand{\balpha}{\bm{\alpha}}    % alpha, vec
\newcommand{\bphi}{\bm{\phi}}    % phi, vec
\newcommand{\bpsi}{\bm{\psi}}    % psi, vec
\newcommand{\bxi}{\bm{\xi}}    % xi, vec
\newcommand{\btheta}{\bm{\theta}}    % theta, vec
\newcommand{\bmu}{\bm{\mu}}    % mu, vec

\newcommand{\bLambda}{\bm{\Lambda}}    % Sigma, mat
\newcommand{\bSigma}{\bm{\Sigma}}    % Sigma, mat

\newcommand{\cF}{\mathcal{F}}  
\newcommand{\cL}{\mathcal{L}}  
\newcommand{\cX}{\mathcal{X}}  
\newcommand{\cY}{\mathcal{Y}}  

\newcommand{\rvx}{\mathsf{x}}    % x, r.v.
\newcommand{\rvy}{\mathsf{y}}    % y, r.v.
\newcommand{\rvz}{\mathsf{z}}    % z, r.v.
\newcommand{\rvw}{\mathsf{w}}    % w, r.v.
\newcommand{\rvv}{\mathsf{v}}    % v, r.v.
\newcommand{\rvm}{\mathsf{m}}    % m, r.v.
\newcommand{\rvt}{\mathsf{t}}    % t, r.v.
\newcommand{\rvH}{\mathsf{H}}    % H, r.v.
\newcommand{\urvx}{\underline{\mathsf{x}}}    % x, r.v. vec
\newcommand{\urvy}{\underline{\mathsf{y}}}    % y, r.v. vec
\newcommand{\urvz}{\underline{\mathsf{z}}}    % z, r.v. vec
\newcommand{\urvw}{\underline{\mathsf{w}}}    % w, r.v. vec
\newcommand{\urvt}{\underline{\mathsf{t}}}    % t, r.v. vec


\newcommand{\defeq}{\triangleq} %\coloneqq
\newcommand{\reals}{\mathbb{R}}
\newcommand{\T}{\mathrm{T}}    % transpose
\newcommand{\F}{\mathrm{F}}    % Frobenius
\newcommand{\BLS}{\mathrm{BLS}}    % BLS
\newcommand{\LLS}{\mathrm{LLS}}    % LLS
\newcommand{\MVU}{\mathrm{MVU}}    % MVU
\newcommand{\dd}{\mathrm{d}}  

\DeclareMathOperator*{\maximize}{maximize}    % maximize
\DeclareMathOperator*{\minimize}{minimize}    % minimize
\newcommand{\st}{\mathrm{subject~to}}    % minimize

% \newcommand{\E}[1]{\mathbb{E}\left[{#1}\right]}
% \newcommand{\Prob}[1]{\mathbb{P}\left({#1}\right)}
\DeclareMathOperator*{\argmax}{arg\,max}
\DeclareMathOperator*{\argmin}{arg\,min}
\DeclareMathOperator*{\argsup}{arg\,sup}
\DeclareMathOperator*{\arginf}{arg\,inf}
\DeclareMathOperator{\diag}{diag}
\DeclareMathOperator{\tr}{tr}
\DeclareMathOperator{\Cov}{Cov}
\DeclareMathOperator{\var}{var}
\DeclareMathOperator{\cov}{cov}
\DeclareMathOperator{\MSE}{MSE}
\DeclareMathOperator{\1}{\mathds{1}} % dsfont required
\DeclareMathOperator{\E}{\mathbb{E}}
\DeclareMathOperator{\Prob}{\mathbb{P}}
\DeclareMathOperator{\im}{im}
\DeclareMathOperator{\rank}{rank}
\DeclareMathOperator{\Bern}{Bern}
\DeclareMathOperator{\Binom}{Binom}

\newcommand\independent{\protect\mathpalette{\protect\independenT}{\perp}}
\def\independenT#1#2{\mathrel{\rlap{$#1#2$}\mkern2mu{#1#2}}}

%%% Local Variables:
%%% mode: latex
%%% TeX-master: "ithw"
%%% End:


\lstset{
  style              = Matlab-editor,
  captionpos         =b,
  basicstyle         = \mlttfamily,
  escapechar         = ",
  mlshowsectionrules = true,
}
\begin{document}
\courseheader



\newcounter{hwcnt}
\setcounter{hwcnt}{5} % set to the times of Homework

\begin{center}
  \underline{\bf Homework \thehwcnt} \\
\end{center}
\begin{flushleft}
  \textcolor{gray}{HANMO CHEN}\hfill
  \today
\end{flushleft}
\hrule

\vspace{2em}
\setlist[enumerate,1]{label=\thehwcnt.\arabic*.}
\setlist[enumerate,2]{label=(\alph*)}
\setlist[enumerate,3]{label=\roman*.}
\setlist[enumerate,4]{label=\greek*)}


\newcommand{\EX}{\mathbb{E}}
\newcommand{\x}{\mathsf{x}}
\newcommand{\y}{\mathsf{y}}

\flushleft
\rule{\textwidth}{1pt}
\begin{itemize}
\item {\bf Acknowledgments: \/} 
  \textcolor{gray}{For Problem 1.2(b), I refer to some answer on StackExchange \small{\url{https://math.stackexchange.com/a/108303}}}
  % \textcolor{gray}{This template takes some materials from course CSE 547/Stat 548 of Washington University: \small{\url{https://courses.cs.washington.edu/courses/cse547/17sp/index.html}}.}

  % \textcolor{red}{If you refer to other materials in your homework, please list here.}
\item {\bf Collaborators: \/}
  \textcolor{gray}{I finish this homework all by myself.} 
% \textcolor{red}{If you finish your homework all by yourself, make a similar statement. If you get help from others in finishing your homework, state like this:}
%   \textcolor{gray}{
%   \begin{itemize}
%   \item 1.2 (b) was solved with the help from \underline{\hspace{3em}}.
%   \item Discussion with \underline{\hspace{3em}} helped me finishing 1.3.
%   \end{itemize}
% }
\item  \emph{I certify that all solutions are entirely in my words and that I have not looked at another student's solutions. I have credited all external sources in this write up.}


  \framebox[\linewidth]{\rule{0pt}{10pt}\textcolor{gray}{\large Hanmo Chen}}
\end{itemize}
\rule{\textwidth}{1pt}


\vspace{2em}

% You may use \texttt{enumerate} to generate answers for each question:

\begin{enumerate}
  \setlength{\itemsep}{3\parskip}

    \item \textit{Cramer-Rao inequality with a bias term.}
    
    \begin{proof}

        The bias $b(x) = \E\left[\hat{x}(\y)\right] -x = \int_{\mathbb{R}} \hat{x}(y) f(y;x)dy - x$. So $b'(x) =  \int_{\mathbb{R}} \hat{x}(y) \frac{\partial f(y;x)}{\partial x }dy -1 $

        Notice that 

        \begin{equation}
          \begin{aligned}
            \E\left[(\hat{x}(\y)-x)^2\right] - b^2(x) & = \E\left[(\hat{x}(\y)-x)^2\right] - \left(\E\left[\hat{x}(\y)-x\right]\right)^2 \\ &= \var(\hat{x}(\y)-x) = \var(\hat{x}(\y))
          \end{aligned}
        \end{equation}

        So the origincal inequality is equivalent to the following one,

        \begin{equation}
          \var(\hat{x}(\y)) \geqslant \frac{[1+b'(x)]^2}{J_{\y}(x)}
        \end{equation}

        where $J_{\y}(x) = \E \left[\left( \frac{\partial}{\partial x} \ln f(y;x)\right)^2\right] $.

          \begin{equation}
            \begin{aligned}
              1+b'(x) = \int_{\mathbb{R}} \hat{x}(y) \frac{\partial f(y;x)}{\partial x } dy= \int_{\mathbb{R}} \hat{x}(y) \frac{\frac{\partial f(y;x)}{\partial x }}{ f(y;x)} f(y;x) dy
            \end{aligned}
          \end{equation}

          Notice that 

          \begin{equation}
            \frac{\partial}{\partial x} \ln f(y;x) =  \frac{\frac{\partial f(y;x)}{\partial x }}{ f(y;x)}
          \end{equation}

          So 

          \begin{equation}
            1+b'(x) = \E\left[\hat{x}(y) \frac{\partial}{\partial x} \ln f(y;x) \right]
          \end{equation}

          And according to the regularity condition,

          \begin{equation}
            \E\left[\frac{\partial}{\partial x} \ln f(y;x)\right] = 0
          \end{equation}

          Thus,
          \begin{equation}
            1+b'(x) = \E\left[\left(\hat{x}(y)-\E[\hat{x}(y)]\right) \frac{\partial}{\partial x} \ln f(y;x) \right]
          \end{equation}

          Using the Cauchy-Schwarz inequality,

          \begin{equation}
            \begin{aligned}
              (1+b'(x))^2 & = \left(\E\left[\left(\hat{x}(y)-\E[\hat{x}(y)]\right) \frac{\partial}{\partial x} \ln f(y;x) \right]\right)^2 \\
              & \leqslant \E\left[\left(\hat{x}(y)-\E[\hat{x}(y)]\right)^2\right] \E\left[\left(\frac{\partial}{\partial x} \ln f(y;x)\right)^2\right]
            \end{aligned}
          \end{equation}

          that is,

          \begin{equation}
            \var(\hat{x}(y)) \geqslant \frac{[1+b'(x)]^2}{J_{\y}(x)}
          \end{equation}

      \end{proof}

        
  \item \begin{enumerate}
    \item Suppose an unbiased estimator $f(y)$ for $x$ exists, so for all $x$ >0,
    \begin{equation}
      \int_{0}^{\frac{1}{x}} xf(y)dy = x \Longrightarrow \int_{0}^{\frac{1}{x}} f(y)dy = 1
    \end{equation}
    That is, $\forall a >0 $

    \begin{equation}
      \int_{0}^{a} f(y)dy = 1
    \end{equation}

    $\forall b>a>0$,

    \begin{equation}
      \int_{a}^{b} f(y)dy = 0 
    \end{equation}

    $f(y) = 0,\forall y >0$, which is not an unbiased estimator. So there is no unbiased estimator for $x$.

    \item First we try to derive an unbiased estimator, $\forall x>0$
    \begin{equation}
      \int_{0}^{x} \frac{1}{x}f(y)dy = x \Longrightarrow \int_{0}^{x} f(y)dy = x^2
    \end{equation}

    Obviously $f(y) = 2y$ is an unbiased estimator. It is easy to show that $y$ is a complete sufficient statistics of $x$, so $\hat{x} (y) = 2y$ is a minimum-variance unbiased estimator for $x$.

  \end{enumerate}

  \item \begin{enumerate}
    \item The distribution function is 
    \begin{equation}
      f(\uy;x) = \left\{\begin{aligned}
        \frac{1}{2\pi}\exp\left({-\frac{(y_1-x)^2}{2}} - \frac{(y_2-x)^2}{2}\right), \quad x>0 \\
        \frac{1}{2\sqrt{2}\pi }\exp\left({-\frac{(y_1-x)^2}{2}} - \frac{(y_2-x)^2}{4}\right), \quad x<0 
      \end{aligned}\right.
    \end{equation}

    \begin{equation}
      \frac{\partial \ln f(\uy;x)}{\partial x} = \left\{\begin{aligned}
        y_1+y_2 -2x, \quad x>0 \\
        y_1+\frac{y_2}{2} - \frac{3}{2} x, \quad x<0 
      \end{aligned}\right.
    \end{equation}
    \begin{equation}
      \begin{aligned}
        J_{\urvy}(x) &= \E\left[\left(\frac{\partial \ln f(\uy;x)}{\partial x}\right)^2\right] = \left\{\begin{aligned}
          2, \quad x>0 \\ 
          \frac{3}{2},\quad x<0 
        \end{aligned}\right.
      \end{aligned}
    \end{equation}
  
    \item Consider the following estimators,
    \begin{equation}
      \begin{aligned}
        \hat{x}_1 (\urvy) = \frac{1}{2} \rvy_1 + \frac{1}{2} \rvy_2 \\
      \hat{x}_2 (\urvy) = \frac{2}{3} \rvy_1 + \frac{1}{3} \rvy_2
      \end{aligned}
    \end{equation}
  
    It is easy to show that $\hat{x}_1 (\urvy),\hat{x}_2 (\urvy)$ are unbiased estimators, and 
  
    \begin{equation}
      \begin{aligned}
        \var(\hat{x}_1 (\urvy)) = \left\{\begin{aligned}
          \frac{1}{2}, x>0 \\
          \frac{1}{4}, x<0 
        \end{aligned}\right. \\
        \var(\hat{x}_2 (\urvy)) = \left\{\begin{aligned}
          \frac{5}{9}, x>0 \\
          \frac{2}{3}, x<0 
        \end{aligned}\right.
      \end{aligned}
    \end{equation}
  
    For $x>0$, $\hat{x}_1 (\urvy)$ achieves the Cramer-Rao Lower Bound and $x<0$, $\hat{x}_2 (\urvy)$ achieves the Cramer-Rao Lower Bound. So there is no minimal-variance unbiased estimator for $x$.
  
  \end{enumerate}

  \item \begin{enumerate}
    \item \begin{equation}
      P_{\urvy}(\uy;x) = x^{y_1+ y_2} (1-x)^{2-y_1-y_2} = (1-x)^2 (\frac{x}{1-x})^{t(\uy)} = a(t(\uy),x)b(x)
    \end{equation}

    So $t(\uy) = y_1+y_2$ is a sufficient statistics for $x$.

    \item \begin{equation}
      \MSE_{\hat{x}}(x) = \E \left[(x-\hat{x}(\uy))^2\right] = x(1-x)
    \end{equation}

    \item \begin{enumerate}
      \item Because $t(\uy)$ is a sufficient statistics, $P_{\urvy}(\uy;x) = P_{\urvy|T}(\uy|t) P_{T}(t;x)$. So $P_{\urvy|T}(\uy|t)$ is independent of $x$. And $x'(t) = \E[\hat{x}(\urvy) | \rvt = t]$ does not depend on $x$.
      \begin{equation}
        x'(t) = \E[y_1| \rvt = t] = \left\{\begin{aligned}
          0, t=0 \\ \frac{1}{2}, t= 1 \\ 1, t= 2
        \end{aligned}\right.
      \end{equation}
      \item $x'(t) = \frac{1}{2} t = \frac{y_1+y_2}{2}$. \begin{equation}
      \MSE_{\hat{x}'}(x) = \E \left[(x-\frac{y_1+y_2}{2})^2\right] = \frac{2}{2}x(1-x)
    \end{equation}

    So $MSE_{\hat{x}'}(x) =  \frac{1}{2} MSE_{\hat{x}}(x)$
    \end{enumerate}

    \item \begin{enumerate}
      \item Because $t(\uy)$ is a sufficient statistics, $P_{\urvy}(\uy;x) = P_{\urvy|T}(\uy|t) P_{T}(t;x)$. So $P_{\urvy|T}(\uy|t)$ is independent of $x$. And $x'(t) = \E[\hat{x}(\uy) | \rvt = t]$ does not depend on $x$.
      \item \textit{(Rao-Blackwell Theorem)} Because the cost function $C(x,\hat{x})$ is convex in $\hat{x}$, using Jensen's inequality, $\forall t$
      \begin{equation}
        C(x,\hat{x}'(t))  = C(x, \E[\hat{x}(\urvy) | \rvt = t]) \leqslant \E\left[C(x,\hat{x}(\urvy)) \mid  \rvt = t \right] 
      \end{equation}

      Take expections of $\rvt$ in the inequality above,

      \begin{equation}
        \E[C(x,\hat{x}'(\rvt))] \leqslant \E\left[C(x,\hat{x}(\urvy)) \right] 
      \end{equation}
    \end{enumerate}
  \end{enumerate}

  \item First we prove it is a sufficient statistics.
  \begin{equation}
    p_{\rvy}(y;x) = \frac{p_{\rvy}(y;H_0)}{p_{\rvy}(y;H_1)} \frac{p_{\rvy}(y;x)}{p_{\rvy}(y;H_0)} p_{\rvy}(y;H_1) = a(t(y),x)b(y)
  \end{equation}

  \begin{equation}
    \begin{aligned}
      &a(t(y),x) =  \frac{p_{\rvy}(y;H_0)}{p_{\rvy}(y;H_1)} \frac{p_{\rvy}(y;x)}{p_{\rvy}(y;H_0)} = \left\{\begin{aligned}
        1,x=H_1 \\ t(y),x=H_0
      \end{aligned}\right. \\ 
      &b(y) = p_{\rvy}(y;H_1)
    \end{aligned}
  \end{equation}

  So $t(y)$ is a sufficient statistics.

  Notice that the distribution function belongs to an exponential family,

  \begin{equation}
    \begin{aligned}
      p_{\rvy}(y;x) & = p_{\rvy}(y;H_1) \exp\left(\1(x=H_0) \log \frac{p_{\rvy}(y;H_0)}{p_{\rvy}(y;H_1)}\right) \\
      & = h(y) \exp\left( w(x) t(y) \right)
    \end{aligned}
  \end{equation}

  Thus $t(y)$ is a complete statistics.

  So $t(y)$ is a complete sufficient statistics
  
  \item \begin{equation}
    \hat{x}_{MAP}(y) = \argmax_{a} p_{\rvx|\rvy} (a|y) = \argmax_{a}  {p_{\rvx,\rvy} (a,y)} = \argmax_{a}  {p_{\rvy|\rvx} (y|a) p_{\rvx}(a)}
  \end{equation}

  Suppose $\rvz$ is the complete data,

  \begin{equation}
    L(x) = \log {p_{\rvy|\rvx} (y|x) p_{\rvx}(x)} = \log p_{\rvz}(z|x) - \log p_{\rvz | \rvy,\rvx} (z|y,x) + \log p_{\rvx}(x)
  \end{equation}

  Take expections of both sides over $p_{\rvz | \rvy,\rvx} (z|y,x')$

  \begin{equation}
    LHS =  \log {p_{\rvy|\rvx} (y|x) p_{\rvx}(x)} 
  \end{equation}

  \begin{equation}
    \begin{aligned}
      RHS = \sum_{z'} p_{\rvz | \rvy,\rvx} (z'|y,x') \log p_{\rvz}(z'|x) - \sum_{z'} p_{\rvz | \rvy,\rvx} (z'|y,x') \log p_{\rvz | \rvy,\rvx} (z'|y,x) + \log p_{\rvx}(x) 
    \end{aligned}
  \end{equation}

  Denote $U(x,x') =  \sum_{z'} p_{\rvz | \rvy,\rvx} (z'|y,x') \log p_{\rvz}(z'|x)+ \log p_{\rvx}(x)$ and $V(x,x') = - \sum_{z'} p_{\rvz | \rvy,\rvx} (z'|y,x') \log p_{\rvz | \rvy,\rvx} (z'|y,x)$ so 

  \begin{equation}
    \log {p_{\rvy|\rvx} (y|x) p_{\rvx}(x)}  = U(x,x') + V(x,x')
  \end{equation} 

  By the same method in lecture, we have 

  \begin{equation}
    V(x,x') - V(x',x') = \sum_{z} p_{\rvz | \rvy,\rvx} (z|y,x') \log \frac{p_{\rvz | \rvy,\rvx} (z|y,x')}{p_{\rvz | \rvy,\rvx} (z|y,x)} = D(p_{\rvz | \rvy,\rvx'} \| p_{\rvz | \rvy,\rvx}) \geqslant 0
  \end{equation}  

  So if we want to find $L(x)  > L(x')$, we just need to make sure $U(x,x') > U(x',x')$.

  The EM-MAP algorithm is as follows.

  \begin{enumerate}
    \item Initialize: choose $x^{(0)}$
    \item Repeat until convergence:\begin{enumerate}
      \item E-step: given previous $x^{(n)}$, compute \begin{equation}
        U(x,x^{(n)}) = \E_{p_{\rvz|\rvy,\rvx}}\left[\log p_{\rvz}(z|x) \big| \rvy = y, \rvx  = x^{(n) }\right] + \log p_{\rvx}(x)
      \end{equation}
      \item M-step: determine $x^{(n+1)}$, \begin{equation}
        x^{(n+1)} = \argmax_{x} U(x,x^{(n)})
      \end{equation}
    \end{enumerate}
  \end{enumerate}

  \end{enumerate}

\end{document}
%%% Local Variables:
%%% mode: latex
%%% TeX-master: t
%%% End:
