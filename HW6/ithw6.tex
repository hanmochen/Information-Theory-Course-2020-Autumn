% Homework template for Information Theory and Statistical Learning
% by Xiangxiang Xu <xiangxiangxu.thu@gmail.com>
% LAST UPDATE: Oct 3, 2019
\documentclass[a4paper]{article}
\usepackage[T1]{fontenc}
\usepackage{amsmath, amssymb, amsthm}
% amsmath: equation*, amssymb: mathbb, amsthm: proof
\usepackage{moreenum}
\usepackage{mathtools}
\usepackage{url}
\usepackage{enumitem}
\usepackage{bm}
\usepackage{graphicx}
\usepackage{subcaption}
\usepackage{booktabs} % toprule
\usepackage[mathcal]{eucal}
\usepackage{dsfont}
\usepackage[numbered,framed]{matlab-prettifier}
%% Definitions for Information Theory & Statistical Learning
%% UPDATED: Sep 30, 2019 by Xiangxiang 
\newcommand{\theterm}{Fall 2020}

\newcommand{\thecoursenameshort}{\textsc{Information Theory and Statistical Learning}}
\newcommand{\thecoursename}{
Tsinghua-Berkeley Shenzhen Institute\\
%\vspace*{0.1in}
\thecoursenameshort
}

\newcommand{\courseheader}{
\vspace*{-1in}
\begin{center}
\thecoursename \\
\theterm
\vspace*{0.1in}
\hrule
\end{center}
}
\newcommand{\uc}{\underline{c}}    % c, vec
\newcommand{\uv}{\underline{v}}    % x, vec
\newcommand{\uw}{\underline{w}}    % w, vec
\newcommand{\ux}{\underline{x}}    % x, vec
\newcommand{\uy}{\underline{y}}    % y, vec
\newcommand{\uz}{\underline{z}}    % z, vec
\newcommand{\um}{\underline{m}}    % m, vec
\newcommand{\ut}{\underline{t}}    % t, vec

\newcommand{\bA}{\mathbf{A}}    % A, mat
\newcommand{\bI}{\mathbf{I}}    % A, mat
\newcommand{\bN}{\mathbf{N}}    % n, mat
\newcommand{\bT}{\mathbf{T}}    % T, mat
\newcommand{\bU}{\mathbf{U}}    % U, mat
\newcommand{\bV}{\mathbf{V}}    % V, mat
\newcommand{\bQ}{\mathbf{Q}}    % Q, mat
\newcommand{\bX}{\mathbf{X}}    % X, mat

\newcommand{\bc}{\bm{c}}    % c, vec
\newcommand{\be}{\bm{e}}    % e, vec
\newcommand{\bu}{\bm{u}}    % u, vec
\newcommand{\bv}{\bm{v}}    % v, vec
\newcommand{\bw}{\bm{w}}    % w, vec
\newcommand{\bt}{\bm{t}}    % t, vec
\newcommand{\bx}{\bm{x}}    % x, vec
\newcommand{\by}{\bm{y}}    % y, vec
\newcommand{\bz}{\bm{z}}    % z, vec

\newcommand{\phib}{\bm{\phi}}    % phi, vec
\newcommand{\psib}{\bm{\psi}}    % psi, vec
\newcommand{\dtm}{\mathbf{B}}    %
\newcommand{\dtmt}{\tilde{\dtm}}    %
\newcommand{\Ab}{\mathbf{A}}    % A mat
\newcommand{\Kb}{\mathbf{K}}    % K mat
\newcommand{\Ib}{\mathbf{I}}    % I mat



\newcommand{\rvby}{\bm{\mathsf{y}}}    % y, rv. vec
\newcommand{\rvbx}{\bm{\mathsf{x}}}    % x, rv. vec
% \newcommand{\bm}{\bm{m}}    % m, vec
\newcommand{\bzero}{\bm{0}}    % 0, vec

\newcommand{\balpha}{\bm{\alpha}}    % alpha, vec
\newcommand{\bphi}{\bm{\phi}}    % phi, vec
\newcommand{\bpsi}{\bm{\psi}}    % psi, vec
\newcommand{\bxi}{\bm{\xi}}    % xi, vec
\newcommand{\btheta}{\bm{\theta}}    % theta, vec
\newcommand{\bmu}{\bm{\mu}}    % mu, vec

\newcommand{\bLambda}{\bm{\Lambda}}    % Sigma, mat
\newcommand{\bSigma}{\bm{\Sigma}}    % Sigma, mat

\newcommand{\cF}{\mathcal{F}}  
\newcommand{\cL}{\mathcal{L}}  
\newcommand{\cX}{\mathcal{X}}  
\newcommand{\cY}{\mathcal{Y}}  

\newcommand{\rvx}{\mathsf{x}}    % x, r.v.
\newcommand{\rvy}{\mathsf{y}}    % y, r.v.
\newcommand{\rvz}{\mathsf{z}}    % z, r.v.
\newcommand{\rvw}{\mathsf{w}}    % w, r.v.
\newcommand{\rvv}{\mathsf{v}}    % v, r.v.
\newcommand{\rvm}{\mathsf{m}}    % m, r.v.
\newcommand{\rvt}{\mathsf{t}}    % t, r.v.
\newcommand{\rvH}{\mathsf{H}}    % H, r.v.
\newcommand{\urvx}{\underline{\mathsf{x}}}    % x, r.v. vec
\newcommand{\urvy}{\underline{\mathsf{y}}}    % y, r.v. vec
\newcommand{\urvz}{\underline{\mathsf{z}}}    % z, r.v. vec
\newcommand{\urvw}{\underline{\mathsf{w}}}    % w, r.v. vec
\newcommand{\urvt}{\underline{\mathsf{t}}}    % t, r.v. vec


\newcommand{\defeq}{\triangleq} %\coloneqq
\newcommand{\reals}{\mathbb{R}}
\newcommand{\T}{\mathrm{T}}    % transpose
\newcommand{\F}{\mathrm{F}}    % Frobenius
\newcommand{\BLS}{\mathrm{BLS}}    % BLS
\newcommand{\LLS}{\mathrm{LLS}}    % LLS
\newcommand{\MVU}{\mathrm{MVU}}    % MVU
\newcommand{\dd}{\mathrm{d}}  

\DeclareMathOperator*{\maximize}{maximize}    % maximize
\DeclareMathOperator*{\minimize}{minimize}    % minimize
\newcommand{\st}{\mathrm{subject~to}}    % minimize

% \newcommand{\E}[1]{\mathbb{E}\left[{#1}\right]}
% \newcommand{\Prob}[1]{\mathbb{P}\left({#1}\right)}
\DeclareMathOperator*{\argmax}{arg\,max}
\DeclareMathOperator*{\argmin}{arg\,min}
\DeclareMathOperator*{\argsup}{arg\,sup}
\DeclareMathOperator*{\arginf}{arg\,inf}
\DeclareMathOperator{\diag}{diag}
\DeclareMathOperator{\tr}{tr}
\DeclareMathOperator{\Cov}{Cov}
\DeclareMathOperator{\var}{var}
\DeclareMathOperator{\cov}{cov}
\DeclareMathOperator{\MSE}{MSE}
\DeclareMathOperator{\1}{\mathds{1}} % dsfont required
\DeclareMathOperator{\E}{\mathbb{E}}
\DeclareMathOperator{\Prob}{\mathbb{P}}
\DeclareMathOperator{\im}{im}
\DeclareMathOperator{\rank}{rank}
\DeclareMathOperator{\Bern}{Bern}
\DeclareMathOperator{\Binom}{Binom}

\newcommand\independent{\protect\mathpalette{\protect\independenT}{\perp}}
\def\independenT#1#2{\mathrel{\rlap{$#1#2$}\mkern2mu{#1#2}}}

%%% Local Variables:
%%% mode: latex
%%% TeX-master: "ithw"
%%% End:


\lstset{
  style              = Matlab-editor,
  captionpos         =b,
  basicstyle         = \mlttfamily,
  escapechar         = ",
  mlshowsectionrules = true,
}
\begin{document}
\courseheader



\newcounter{hwcnt}
\setcounter{hwcnt}{6} % set to the times of Homework

\begin{center}
  \underline{\bf Homework \thehwcnt} \\
\end{center}
\begin{flushleft}
  \textcolor{gray}{HANMO CHEN}\hfill
  \today
\end{flushleft}
\hrule

\vspace{2em}
\setlist[enumerate,1]{label=\thehwcnt.\arabic*.}
\setlist[enumerate,2]{label=(\alph*)}
\setlist[enumerate,3]{label=\roman*.}
\setlist[enumerate,4]{label=\greek*)}


\newcommand{\rvs}{\mathsf{s}} 
\newcommand{\rvr}{\mathsf{r}} 
\newcommand{\bB}{\mathbf{B}}  
\newcommand{\bg}{\boldsymbol{g}} 
\newcommand{\Bf}{\boldsymbol{f}} 

\flushleft
\rule{\textwidth}{1pt}
\begin{itemize}
\item {\bf Acknowledgments: \/} 
  \textcolor{gray}{For $L_2$-norm I refer to the wikipedia \url{https://en.wikipedia.org/wiki/Matrix_norm}. For 6.4, I refer to \url{https://ieeexplore.ieee.org/document/8849720}\footnote{ S. Huang, X. Xu, L. Zheng and G. W. Wornell, "An Information Theoretic Interpretation to Deep Neural Networks," 2019 IEEE International Symposium on Information Theory (ISIT), Paris, France, 2019, pp. 1984-1988, doi: 10.1109/ISIT.2019.8849720.}}
  % \textcolor{gray}{This template takes some materials from course CSE 547/Stat 548 of Washington University: \small{\url{https://courses.cs.washington.edu/courses/cse547/17sp/index.html}}.}

  % \textcolor{red}{If you refer to other materials in your homework, please list here.}
\item {\bf Collaborators: \/}
  \textcolor{gray}{I finish this homework all by myself.} 
% \textcolor{red}{If you finish your homework all by yourself, make a similar statement. If you get help from others in finishing your homework, state like this:}
%   \textcolor{gray}{
%   \begin{itemize}
%   \item 1.2 (b) was solved with the help from \underline{\hspace{3em}}.
%   \item Discussion with \underline{\hspace{3em}} helped me finishing 1.3.
%   \end{itemize}
% }
\item  \emph{I certify that all solutions are entirely in my words and that I have not looked at another student's solutions. I have credited all external sources in this write up.}


  \framebox[\linewidth]{\rule{0pt}{10pt}\textcolor{gray}{\large Hanmo Chen}}
\end{itemize}
\rule{\textwidth}{1pt}


\vspace{2em}

% You may use \texttt{enumerate} to generate answers for each question:

\begin{enumerate}
  \setlength{\itemsep}{3\parskip}

  \item \begin{enumerate}
    \item \begin{equation}
      p_{\rvy_1\rvy_2}(y_1,y_2;x) = 1, x\leqslant y_1,y_2 \leqslant x+1
    \end{equation} 
    And $p(\rvs=\max(\rvy_1,\rvy_2) \leqslant s)) = (s-x)^2, x \leqslant s \leqslant x+1$, so \begin{equation}
      p_{\rvs}(s;x) = 2(s-x), x \leqslant s \leqslant x+1
    \end{equation}
    \begin{equation}
      \begin{aligned}
        \frac{p_{\rvy_1\rvy_2}(y_1,y_2;x)}{p_{\rvs}(s;x)} = \frac{1}{2(s-x)}, x \leqslant y_1,y_2,s \leqslant x+1
      \end{aligned}
    \end{equation}

    So $\rvs$ is not a sufficient statistic for $p_{\rvy_1\rvy_2}(y_1,y_2;x)$.

  \item $p_{\rvr}(r;x) = p_{\rvr}(-r;x) $ due to symmetry. Suppose $r\geqslant 0$, 
   \begin{equation}
    p_{\rvr} (r;x) = \int_{x}^{x+1} p_{\rvy_1}(y+r;x) p_{\rvy_2}(y;x) dy = 1-r
  \end{equation}

  So $p_{\rvr} (r;x) = 1-|r|$ does not depend on $x$, $\rvr$ is an ancillary statistic for $p_{\rvy_1\rvy_2}(y_1,y_2;x)$.

  \item Yes. Because we can recover $\rvy_1,\rvy_2$, with $\rvs,\rvr$.
  \begin{equation}
    (\rvy_1,\rvy_2) = \left\{\begin{aligned}
      (\rvs,\rvs-\rvr), \quad r \geqslant 0 \\
      (\rvs+\rvr,\rvs), \quad r <0
    \end{aligned}\right.
  \end{equation}

  So $\boldsymbol{\mathsf{u}} = [\rvs,\rvr]^T $ is a sufficient statistic for $p_{\rvy_1\rvy_2}(y_1,y_2;x)$ and 
  \end{enumerate}

  \item Let $\rvx=\rvy$ be randomly chosen from $\{0,1,2\}$ with equal probability $\frac{1}{3}$ and $\xi(\rvx)=\rvx,\eta(\rvy) = \rvy^2$. $\rho(\rvx,\rvy) =\rho(\rvx,\rvx) =1$. \begin{equation}
    \rho(\xi(\rvx),\eta(\rvy)) = \rho(\rvx,\rvx^2) = \frac{\E[\rvx^3] - \E[\rvx^2]\E[\rvx]}{\sqrt{\var(\rvx)\var(\rvx^2)}} = \sqrt{\frac{12}{13}} \neq \rho(\rvx,\rvy)
  \end{equation}

  \item \begin{enumerate}
    \item \begin{enumerate}
      \item \begin{equation}
        \begin{aligned}
          \bpsi^T \bB \bphi & =  \sum_{x\in \cX,y \in \cY} g(y) \sqrt{P_{\rvy}(y) } \frac{P_{\rvx\rvy}(x,y)}{\sqrt{P_{\rvx}(x)P_{\rvy}(y)}} f(x) \sqrt{P_{\rvx}(x)} \\
          & = \sum_{x\in \cX,y \in \cY} f(x) g(y) P_{\rvx\rvy}(x,y) \\
          & = \E[f(x)g(y)]
        \end{aligned}
      \end{equation}
      \item \begin{equation}
        \begin{aligned}
          \bB \bphi(y) & = \sum_{x\in \cX} \frac{P_{\rvx\rvy}(x,y)}{\sqrt{P_{\rvy}(y)}} f(x) \\
          & =  \sqrt{P_{\rvy}(y)} \sum_{x\in \cX} P_{\rvx | \rvy}(x|y) f(x) \\
          & = \sqrt{P_{\rvy}(y)} \E[f(\rvx) | y]
        \end{aligned}
      \end{equation}
      So $\bB \bphi \leftrightarrow \E[f(\rvx) | \rvy]$
      \item  \begin{equation}
        \begin{aligned}
          \bB^T \bpsi(x) & = \sum_{y\in \cY} \frac{P_{\rvx\rvy}(x,y)}{\sqrt{P_{\rvx}(x)}} g(y) \\
          & =  \sqrt{P_{\rvx}(x)} \sum_{y\in \cY} P_{\rvy | \rvx}(y|x) g(y) \\
          & = \sqrt{P_{\rvx}(x)} \E[g(\rvy) | x]
        \end{aligned}
      \end{equation}
      So $\bB^T \bpsi \leftrightarrow \E[g(\rvy) | \rvx]$
    \end{enumerate}

    \item \begin{equation}
      \bB \bphi_1(y) = \sum_{x\in \cX}\frac{P_{\rvx\rvy}(x,y)}{\sqrt{P_{\rvy}(y)}} = \sqrt{P_{\rvy}(y)} 
    \end{equation}

    \begin{equation}
      \bB^T \bpsi_1(x) = \sum_{y\in \cY}\frac{P_{\rvx\rvy}(x,y)}{\sqrt{P_{\rvx}(x)}} = \sqrt{P_{\rvx}(x)} 
    \end{equation}

    So $\bB\bphi_1 = \bpsi_1, \bB^T\bpsi_1 = \bphi_1$. Its means that if we let $f(x)=g(y)=1$ in (a), we have $\E[1|\rvy] = \E[1|\rvx] = 1$

    \item  The $L_2$-norm of $\bB$ is equivalent to 
    \begin{equation}
      \| \bB \|_2  = \sup \| \bB \bx \|_2, \st  \| \bx \|_2 = 1
    \end{equation}

    To maximize $\| \bB \bx \|_2$, every component in $\bx$ must $\geqslant 0 $. Thus for every $\bX \in \mathbb{R}^{|\cX|}$, there is a correspond distribution of $X \in \cX$. So $\bX$ can be expressed as 
    
    \begin{equation}
      \bx = \left[\sqrt{P_{\rvx'}(1)},\sqrt{P_{\rvx'}(2)},\cdots,\sqrt{P_{\rvx'}(|\cX|)}\right]^T
    \end{equation}

    Thus 

    \begin{equation}
      \begin{aligned}
        \| \bB \bx \|^2_2 & = \sum_{y \in \cY} \left(\sum_{x\in \cX} \frac{P_{\rvx\rvy}(x,y)}{\sqrt{P_{\rvx}(x)P_{\rvy}(y)}} \sqrt{P_{\rvx'}(x)} \right)^2  \\
        & = \sum_{y \in \cY}P_{\rvy}(y)  \left(\sum_{x\in \cX} \frac{P_{\rvx|\rvy}(x|y)}{\sqrt{P_{\rvx}(x)}} \sqrt{P_{\rvx'}(x)} \right)^2 \\
        & = \sum_{y \in \cY}P_{\rvy}(y)  \E^2\left[\frac{\sqrt{P_{\rvx'}(x)}}{\sqrt{P_{\rvx}(x)}} \big| y \right] \\
        & \leqslant \sum_{y \in \cY}P_{\rvy}(y) \E\left[\frac{P_{\rvx'}(x)}{P_{\rvx}(x)} \big| y \right] = \E \left[\E\left[\frac{P_{\rvx'}(x)}{P_{\rvx}(x)} \big| y \right] \right] \\
        & \quad = \E [\frac{P_{\rvx'}(x)}{P_{\rvx}(x)}] = 1
      \end{aligned}
    \end{equation}

    And the "=" holds when $\bx = \bphi_1$.  So $\|\bB\|_2 = 1$
  \end{enumerate}

  \item \begin{enumerate}
    \item The empirical mean $\frac{1}{n} \sum_{i=1}^n \log Q_{\rvy | \rvx} (y_i | x_i ) $ is the realization of expectation $ \E_{P_{\rvx\rvy}}[\log Q_{\rvy | \rvx}(x,y)]$. \begin{equation}
      \begin{aligned}
        (\bg^*,b^*) & = \argmax_{\bg,b} D(P_{\rvx\rvy}\| P_{\rvx}Q_{\rvy|\rvx} ) \\
        & = \argmax_{\bg,b} \E_{P_{\rvx\rvy}} \left[ \log \frac{P_{\rvx\rvy}(x,y)}{P_{\rvx}(x)Q_{\rvy|\rvx}(y|x)}\right] \\
        & = \argmin_{\bg,b} \E_{P_{\rvx\rvy}}[\log Q_{\rvy | \rvx}(y|x)] \\
        & = \argmin_{\bg,b} \frac{1}{n} \sum_{i=1}^n \log Q_{\rvy | \rvx} (y_i | x_i )
      \end{aligned}
    \end{equation}

    \item First, we will prove the fisrt order approximation of K-L divergence, i.e. if $P_1(x) -P_2(x) = O(\varepsilon) $,
    
    \begin{equation}
      \begin{aligned}
        D(P_1(x) \| P_2(x)) & = -\sum_{x \in \cX} P_1(x) \log \left(\frac{P_1(x) + P_2(x) -P_1(x)}{P_1(x)}\right) \\
        & = - \sum_{x\in \cX} P_1(x) \left( \frac{P_2(x)-P_1(x)}{P_1(x)} -\frac{1}{2} \frac{(P_2(x)-P_1(x))^2}{(P_1(x))^2} + O(\epsilon^3)\right) \\ 
        & \approx \frac{1}{2} \sum_{x\in \cX} \frac{(P_2(x)-P_1(x))^2}{P_1(x)} = \frac{1}{2} \sum_{x\in \cX} \frac{(P_2(x)-P_1(x))^2}{P_2(x)} \frac{P_2(x)}{P_1(x)}  \\
        & = \frac{1}{2} \sum_{x\in \cX} \frac{(P_2(x)-P_1(x))^2}{P_2(x)}  ( 1- \frac{O(\epsilon)}{P_1(x)}) \\
        & \approx \frac{1}{2} \sum_{x\in \cX} \frac{(P_2(x)-P_1(x))^2}{P_2(x)} 
      \end{aligned}
    \end{equation}

    Then we just need to prove that $P_{\rvx\rvy}(x,y) - P_{\rvx}(x)Q_{\rvy|\rvx}(y) = O(\epsilon)$.

    Because $\Bf^T(x) \bg^*(y) + d^*(y) = O(\varepsilon)$, 
    \begin{equation}
      P_{\rvy}(y) e^{\Bf^T(x) \bg^*(y) + d^*(y)} =  P_{\rvy}(y) (1+ \Bf^T(x) \bg^*(y) + d^*(y) + O(\varepsilon^2))
    \end{equation}

    \begin{equation}
      \begin{aligned}
        \sum_{y \in \cY}  P_{\rvy}(y) (1+ \Bf^T(x) \bg^*(y) + d^*(y) + O(\varepsilon^2)) = 1 + \E_{P_{\rvy}} [\Bf^T(x) \bg^*(y)] + \E_{P_{\rvy}} [d^*(y)]  + O(\varepsilon^2)
      \end{aligned}
    \end{equation}

    Without loss of generality, we can assume $\E_{P_{\rvy}} [\bg^*(y)] = \E_{P_{\rvy}} [d^*(y)] = 0$.


    \begin{equation}
      \begin{aligned}
        Q_{\rvy | \rvx}[(x,y)] & = \frac{P_{\rvy}(y) (1+ \Bf^T(x) \bg^*(y) + d^*(y) + O(\varepsilon^2))}{ \sum_{y' \in \cY}  P_{\rvy}(y') (1+ \Bf^T(x) \bg^*(y') + d^*(y') + O(\varepsilon^2))} \\
        & = \frac{ P_{\rvy}(y) \left(1+ \Bf^T(x) \bg^*(y) + d^*(y) + O(\varepsilon^2)\right)}{1+ O(\varepsilon^2)} \\
        & = P_{\rvy}(y) \left(1+ \Bf^T(x) \bg^*(y) + d^*(y) + O(\varepsilon^2)\right)(1-O(\varepsilon^2)) \\
        & = P_{\rvy}(y) \left(1+ \Bf^T(x) \bg^*(y) + d^*(y)\right) + O(\varepsilon^2)
      \end{aligned}
    \end{equation}

    Thus,

    \begin{equation}
      P_{\rvx\rvy}(x,y) - P_{\rvx}(x)Q_{\rvy|\rvx}(y) = P_{\rvx}(x) P_{\rvy}(y) + O(\varepsilon) -P_{\rvx}(x)P_{\rvy}(y)(1+O(\varepsilon))  = O(\varepsilon)
    \end{equation}

    So we have,

    \begin{equation}
      D(P_{\rvx\rvy}(x,y) \| P_{\rvx}(x)Q_{\rvy|\rvx}(y) ) \approx \frac{1}{2} \sum_{x\in \cX,y\in \cY} \frac{(P_{\rvx\rvy}(x,y) - P_{\rvx}(x)Q_{\rvy|\rvx}(y))^2}{P_{\rvx}(x)Q_{\rvy|\rvx}(y)}
    \end{equation}
  \end{enumerate}


  \end{enumerate}

\end{document}
%%% Local Variables:
%%% mode: latex
%%% TeX-master: t
%%% End:
